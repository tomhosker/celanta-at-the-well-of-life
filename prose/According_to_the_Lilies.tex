\renewcommand{\thesection}{\Roman{section}} 

\proseepigraph{...and for my thirst they gave me vinegar to drink.}\blindfootnote{The title is taken from Psalm 69 (LXX: 68), as is the epigraph. The author dedicates this story to his father.}

\section{}

How did I meet her? I was living in Pembrokeshire at the time -- that's where I'm from -- and I was working as a cashier at a tourist spot. It was a dairy farm once upon a time; then in 1986 the herd manager decided to turn it into a kind of rustic theme park, adding donkey rides, a fairground, candyfloss, even a gift shop. It was a shrewd decision. She -- the proprietress -- had just come into her inheritance. (There were stories about her father having fallen off of his yacht, but I believe he actually died peacefully in a hotel room in Picadilly.) The old farm happened to be one of her family's holdings. She decided to pay us a visit, inspect the troops, or something like that; that was what we were told. She came down to our section when most of us were on a break, and, because this was the early 1990s and it was a nice day, everyone was outside smoking. Except I wasn't: I've ever been precious about my health or anything; I actually like the smell; I've just never been able to tolerate the sensation of smoke in my throat. She did try to be friendly -- she talked to a couple of us -- but she couldn't stop herself from making a face whenever a puff of smoke drifted towards her nostrils. As she left, she touched me lightly on my right shoulder. I know now that that was the signal for her staff: this is the one I want.

When I turned up for work a couple of days later, my boss asked me if I could work late. I told him yes, and he explained that I had a special task. Upper management were canvassing frontline workers. Best behaviour, and all that. A car arrived at five o'clock to pick me up. The driver's name was George; he told me he had worked for the proprietress' family all his life, in one role or another. He drove me to an old stonebuilt farmhouse -- from the front, it looked like it had six or seven bedrooms -- about fifteen minutes out of Milford Haven. It was a good spot: half a mile from the Coastal Path, but still with a good view of the sea. George showed me into the front room, and there was the proprietress to meet me with Welsh cakes and a pot of coffee.

We talked about nothing for maybe twenty minutes. She told me to call her Ellie. She asked me how I was treated at the farm. I couldn't complain. Then she pointed out that I'd barely touched my coffee since I'd sat down.

`To tell you the truth, I don't really drink coffee. I'm not a Mormon or anything. I just don't like the taste.'

She asked me if I would prefer a proper drink. I said I would, so she told me to follow her. She led me upstairs, into what looked to be her own room.

`Take a seat,' she said. There was nowhere to sit but the bed. `I'm living out of a suitcase at the moment, but I always make sure the fundamentals are taken care of.' She gestured to one corner of the room, in which there was a kind of impromptu bar. `What can I get you?'

`Oh. John Smith's please, if you've got it,' I said, trying to be manly. It was a warm evening. Had she not been there to impress, I would have poured myself a gin and tonic.

`I do have it.' She handed me a can and a pint glass. Then she poured out a bottle of Old Peculier, and, having admired her work for a few moments, took a small sip.

`Forgive me for saying so,' I said, `but you don't often see a woman enjoying that kind of beer.'

She smiled at me. She had an exceptionally warm smile, almost comical, as if her mouth was too big for her face. `My father was a recovering alcoholic,' she said. `When I was little, my mum used to ship me off to stay with him for a few weeks. He had a farm in Normandy. He taught me to drive a tractor. And I used to drink his non-alcholic beer. I was eight years old, and I loved the taste of it straight away.'

She asked me to tell her about myself. I told her there wasn't much to tell. She told me to tell her anyway. So I told her my story: how I had been studying physics at Bristol, how I had got my girlfriend pregnant, how I had dropped out and found a job, how she had had an abortion at twenty-three weeks.

I noticed her widening her eyes in sympathy. The she pulled a book from the shelf nearest the bed and sat down next to me. She opened the book at a specific page and gave it to me; it was a poem.

`I can't read French,' I explained. `If it was Welsh, I might have half a chance, but --'

`That's all right,' she said. `I can translate for you.' She put her finger on the first line: `\textit{Il pleure dans mon coeur} -- It rains in my heart.' I forget the rest.

Once she had finished reading -- I think it was only a dozen or so lines -- she closed the book and put it back. Then she kissed me. When I opened my eyes, I noticed that she was already watching me, and it seemed as if she had been watching my reaction the whole time, like I was an amuse-bouche, one of several, and she was trying to decide which one she liked best. Her eyes were a very vibrant blue, like the waters off the Costa del Sol in the glossy package holiday brochures they had back then. She was sorry, she said, but she had a meeting in London the following morning; she had to leave shortly, and she hadn't yet packed. George would drive me home.

`But one more thing,' she said as I began to get up, and kissed me again.

\section{}

George picked me up after work again the following week, and brought me back to Ellie and her farmhouse. I remember her hugging me as soon as I walked in. We sat on the floor of her room and talked. We drank a bit, but not enough to make me tipsy, and we each said some more about our lives up until then. She tried to teach me how to play canasta. Then at some point she put her head on my shoulder, and we started kissing.

I could feel my soul starting to wrap itself around the image it was forming of her. If that sounds a bit much, bear in mind that I had only been with one woman at that time, and I had only just turned twenty. I unbuttoned my shirt and, taking her right hand, held it against my chest, so that she could feel my heart beating under my skin. She, in turn, held my hand against her blouse, although, as she explained, she had never had much success finding her own heartbeat. I told her that I couldn't quite find it either, but it was a very nice breast all the same. Then she kissed me and made the same excuse as last time -- something about an early start the following morning -- and George drove me home as before.

\section{}

I thought I must have said something to upset Ellie, because I didn't hear from her again for almost two months. But then George appeared in the breakroom one morning just before lunch; I was to go with him to the farmhouse right away. Ellie was standing in the dining room when I arrived. She asked me to take a seat.

She did not seem to be quite herself. She was wearing a skirt and a blazer, but had taken off her shoes, so that only her tights came between her toes and the slate tiles beneath them. She opened her mouth to say something, stopped, then tried again.

`For reasons which I am not willing to divulge at this time,' she said, `it is essential for me have a child within the next eighteen months, and I would like to know whether you would be willing to assist me in achieving this goal.'

`Gosh,' I said. `Forgive me if I've got the wrong end of the stick, but you want the two of us to have a baby together?'

She nodded.

Seeing the agony written over her face, I continued: `I imagine someone's told you by now how babies are made?'

The tension that had been ratcheting up in her chest seemed to ease off by half a click. She said she wanted to do things as quietly and straightforwardly as possible, and natural methods were more conducive to that end than the alternatives.

Did she want anything more from me than a means of conception?

`I would like my child to know his or her biological father,' she said. `But one thing you must understand if we are going to proceed any further: I am not looking for any kind of husband or boyfriend, and I never will be.'

She laid out the terms of the arrangement in more detail. Too much detail, to be honest, but she was talking excitedly like she just had to talk, so I reasoned it was better to let her offload whatever was on her mind. My name would be kept off the birth certificate. I would receive an annuity: nothing spectacular, but several times more than my current income. And I was welcome to stay at the farmhouse -- there was an annexe with several empty bedrooms -- whenever, and for as long as, I liked.

I spoke as soon as there was a gap: `Well,' I said, `today's your lucky day.'

\section{}

I first met Paula in a pub in Rosyth. It was late September, an unseasonably warm afternoon. Ellie had suggested that, before we jumped into the deep end of co-parenthood, we ought to get to know each other first. You get to know people through the things they love, and it turned out that the thing she loved more than anything else in the world was field hockey; she was the centre forward for an ill-fated London team, which had just lost 6-1 in a friendly to Dumfermline Ladies. So there I was at the bar, attempting to drown my boredom with the other players' husbands and sweethearts. I ordered a pint of Tennent's; the barmaid handed it to me, but as I grasped the glass it slipped through my fingers, soaking the bar mat and dripping onto the floor. I felt a hand squeeze my right shoulder.

`Don't worry, old thing. The same thing happened to me the other day. Let me buy you another. I'm Paula, by the way.'

We sat down at a table next to a large window, looking out over the Firth of Forth. The water was calm. It was one of those placid September late afternoons.

`What did you think of the hockey, then?' she asked me.

`Ball games aren't really my thing,' I said, `but there are worse ways of spending a Saturday than watching twenty-two girls in short skirts reenact the Battle of Flodden.'

She smiled indulgently at that remark, then, looking out of the window for a moment, replied, `Well spoken, sir. Except that the English won Flodden, if I remember correctly.'

Either it was my imagination or she gave me a little wink. Then we talked for a bit about the team's chances for the coming season. There was little room for optimism. Ellie was our best player, and it was likely that she would have to miss a lot of the upcoming training sessions, maybe even some matches, given the responsibilities she had acquired with her inheritance. Paula herself had had to step back from the team; a recent injury had rendered her, in her own words, `a pretty atrocious goalie'.

Something I asked caused her to launch into a lengthy digression concerning a recent change to the offside rule, and, instead of listening properly, I began to study her appearance. She was noticeably sturdier than Ellie, but she was quite beautiful in her own way. She had a clever, almost feline-looking face, and sparkling green eyes enhanced by a few touches of mascara. Just as I was beginning to wonder about her age -- thirty, perhaps? -- she got up, and, touching me on the shoulder, said in half a whisper, `You're just as handsome as Ellie said you were. Best of luck to you.'

\section{}

That sounded like a permanent valediction. In fact, Paula came to stay at the farmhouse quite frequently that autumn, as did I. That was how she came to tell me her story. She and Ellie had met in their teens, at an expensive boarding school which had reluctantly gone coeducational. Both were keen to point out that cost was no guarantee of quality. Both had trouble fitting in. Even though Paula was two years older, an eternity at that age, they had become close friends. When Paula failed to get into Oxford -- `Balliol, modern languages,' she announced with a mixture of satisfaction and self-reproach -- she decided to change tack all together. She trained as a marine engineer -- she admitted a lifelong fascination with shafting and high-performance lubricants -- and thereafter had been in the service of several cruise lines. She worked in the Mediterranean mostly. The trips were always four months long, but the spacing between them could be quite irregular -- sometimes a few weeks, sometimes almost a whole year -- depending on the needs of the company, the training courses necessary to maintain her certification, and how she felt. She had had some catastrophic falling out with her biological family; I never got to the bottom of it, but the effect was that, since her job meant that she was out of the country for big chunks of each year, and since she couldn't stay with her parents, she spent her leave living with Ellie more often than not. At the point that I became acquainted with her, she had had a permanent room at the farmhouse for some time.

They lived like a pair of wayward fourteen-year-olds each time they got together: staying up until four in the morning talking and drinking strawberry schnapps; watching the first four Freddy Krueger films in one go; smoking out of bedroom windows; subsisting entirely on jaffa cakes and sliced ham. They loved to tell each other the most appalling stories. One of Ellie's sticks in my mind like a fishbone; she claimed it had come from another another of her old classmates, who had recently qualified as a doctor. An ambitious young solicitor went to a clinic with a particularly unpleasant case of vaginitis. After several excruciating minutes in the stirrups, Ellie's acquaintance managed to pull a used prophylactic from deep within her. Apparently, it had been there at least a month. The patient's poor fianc\'e, meanwhile, was sitting in the waiting area in blissful ignorance. He was last seen getting the hairdryer treatment in the car park.

They were very fond of practical jokes too: a tablespoon of chili flakes in one's breakfast cereal, shampoo bottles topped off with mayonnaise. No marks for originality, but the execution was very smooth. Paula had an inflatable animal -- I forget whether it was Godzilla or a crocodile -- which, if you really blew it up, could get as big as a grown man. It used to turn up in all sorts of places: kitchen cupboards, behind shower curtains, in bed beside you when you woke up. Nor were innocent bystanders such as myself entirely safe from their mischief. For several months, I could not walk into my bedroom in the annexe without feeling that someone was about to jump out at me.

\section{}

So Ellie and Paula got to know me, and I them. Then one Wednesday afternoon -- it was about a fortnight before Christmas -- Paula and myself were alone in the house, because Ellie had had to go to London for a few days to attend to some sort of crisis in one of the family businesses. I was reading by the fire in the main house, a Jilly Cooper novel of all things. It was just starting to get dark, and I went to the bathroom, which caused me to pass by Paula's room. The door was open; she was sitting on the bed with her eyes closed; the colour had gone from her face. I rapped the door as gently as I could manage.

`Is everything okay?' I said. `Sorry, but I couldn't help noticing that you seem to be a bit under the weather.'

`It's fine,' she said, smoothing down her teeshirt. `I was praying.'

`Gosh. Forgive me, but you never struck me as the type.'

`What exactly are you insinuating?'

We both laughed, but the tension was still there.

Then she said, `I'm dying. I was asking if I might have a bit more time.'

`Dear God. Are you in pain?'

`No, no, not at all, sweetheart. It's okay. Ellie already knows; she's known for a while. Everything that can be sorted out has already been sorted out. You don't need to worry.'

`Am I allowed to ask what's wrong with you?'

`I have a neurodegenerative disease. Do you know what that is?'

I told her I did, but that she might have to bear with me if I misunderstood any of the specifics of her condition. Then she told me a bit more about what she had been through. She had her first symptom about a year ago; she was washing her car one winter afternoon, and her right hand had gone numb. At the time, she thought it must just have been the cold; she regained feeling after a couple of hours. But as the weeks passed her grip in that hand became weaker and weaker. Then the weakness spread into her arm.

`You might have noticed,' she said, `that I hardly use my right hand for anything, but I'm actually right-handed.' She got a firm diagnosis six months later. `No one can tell me when I'm going to die. I might last another year. I might last five.'

`Knowing my luck,' I said, `you'll be like that Stephen Hawking, and I'll be wheeling you around for the next thirty years.'

She laughed, thank goodness. We both did.

`Would it help if I stayed with you for a bit,' I said, `or would you prefer to be left alone?'

`No, I'd like you to stay.'

So I sat down next to her on the bed, and we prayed together in silence for about half an hour. I didn't have any specific religious faith at the time, and neither did she. I don't think either of us knew whether we were praying to Jupiter, Jehovah or Quetzalcoatl, but it didn't seem to matter. It just felt like the right thing to do in the circumstances.

\section{}

After Christmas, Ellie moved me out of the annexe and into a cottage a mile and a half away, on the edge of the estate. It was scandalous, Ellie liked to tease me, to keep a young man such as myself under the same roof as two respectable young ladies. I was still a part of their lives. I used to go round to the farmhouse on a weekend for a drink and a catchup. I had a thing about drinking negronis at the time, and Ellie had this fantastic bottle of red vermouth. (I actually couldn't tell you whether it was good quality, but the label was very ornate.) It was always a good time. Both of us knew about Paula's illness, and each of us knew that the other knew, but we would never, and never did, discuss it. But Ellie asked me if I could check on Paula, and perhaps have lunch with her, on those days when business called Ellie back to London, or wherever else she needed to go.

Naturally, I grew closer to Paula during those lunches. She would walk down to my cottage; she insisted on making the most of her legs while they would still do as they were told. My skills as a cook are, to put it charitably, limited, but she seemed happy enough with sandwiches or boiled eggs. Once the weather became a bit more civilised, we used to venture out to the Coastal Path -- whenever I try to imagine the landscape of heaven, I always end up picturing Pembrokeshire in late spring -- but in those first few weeks we used to sit in my tiny front room and talk about whatever was on my mind. After the first couple of times, she would bring a musty old copy of the New World Translation along with her; she had been a Witness in a former life, she explained, or at least her parents were. She would read a few verses from one of the gospels out loud, and then say whatever it was that had touched her heart. Almost always, we would end up talking about something completely different, but it was helpful to have somewhere to start.

Her knowledge of the Bible was actually sort of shaky -- not that mine was any better back then. I can see now that a lot of the time she was just regurgitating undigested lumps of \textit{Watchtower} gobbledegook. Then every so often she would come out with something startlingly insightful, the sort of thing which would make me feel self-conscious that I would never have the spiritual horsepower to come up with anything like that myself. When we were looking at Matthew's version of the Sermon on the Mount, at the point where Jesus tells the multitudes that they are the salt of the earth, she explained that this meant that human beings, having a sophisticated and self-reflective consciousness, are able to perceive the world more fully than any other animal, and thus give the world a flavour for which no other creature could substitute. I didn't have anything clever to say in response to that.

\section{}

Like I said, as the weather got warmer we tended to venture out more and more onto the Coastal Path, and have our conversations there. Nor did we limit ourselves to what was within walking distance of the estate. This was where it became really useful to have George on hand to drive us; he would drop us off in one of those tiny whitewashed seaside villages, and then pick us up a few miles round the coast. (No mobile phones in those days, of course, but he seemed quite content to wait for us in the nearest pub with a newspaper and a half pint of Tetley's.) We were adventurous: Manarfon to Tenby in the south, Fishguard to Strumble Head Lighthouse up by the border with Ceredigion, and a good percentage of the coast and estuary in between. We must have undertaken a few dozen such expeditions that spring and summer, the exact routes for most of which, inevitably, I now struggle to trace on a map.

But the journey from Porthgain to Abercastle is one of the exceptions; there is no fog around that memory. I think -- in fact I know -- that it was her favourite walk -- as it was mine. I've been fascinated by Porthgain ever since that first trip with my parents when I was seven or eight. The place has a post-industrial, almost Audenian, romance to it, with its slipway and ruined quarries and the faint remains of a narrow-gauge railway. And then those four miles of paradise up the coast to Abercastle: the wild seaside flowers, and the little waterfall pouring into the ocean, and the banks of gorse smelling of coconuts and riddled with spiders.

`It's not suffering in and of itself that makes me doubt the existence of a loving God,' Paula announced one afternoon, as we rounded the hill east of Porthgain harbour, `it's the everyday nastiness of life. I can sort of see how the grand tragedies -- my situation, for example, or a young mother who gets diagnosed with brain cancer, or her children get killed in a car crash -- I can sort of see how experiences like that might elevate a person's soul -- although of course you would then ask what kind of supreme being would need to resort to those methods. But what about toothache or tapeworms or dandruff? Why create any of that stuff?'

`I think you might have a point there,' I said. `Forget Darwin and Galileo. How can one reconcile Christianity with the existence of Esther Rantzen?'

That got a chuckle. Then she talked a bit more about the kinds of thoughts she had been thinking.

`It's difficult to reconcile the prognosis they've given me with how I'm feeling right now. I don't feel like I'm dying.' Then she spoke about feeling afraid: `I know I should say that what really bothers me is the effect all this is going to have on Ellie and the others I'll leave behind. And of course I do worry about that, but that's not foremost in my mind. It's lights out, journeys end, ``Good night, Vienna'', timeless oblivion. Not being here, not being anywhere.'

As we cleared the last headland blocking our view of Abercastle, I asked Paula if she would like to visit an ancient monument. She said she would; it was only three hundred yards from the Coastal Path. We stood on the grass with our toecaps touching the wire fence separating the thing from the byway. Unless you know what you're looking for, it just looks like a pile of boulders, about the same dimensions as a Land Rover. It acquired the name which appears on maps today, Carreg Samson, during the Middle Ages, long, long after it was put up. It was likely built, I explained to Paula, around the same time that the Egyptians were building their pyramids; one book says there are probably hundreds of bodies buried underneath it. She closed her eyes and began to recite something from memory:

\begin{quoting}
    But the dead know nothing, and have nothing left to hope for; but the memory of them is lost. All their love and their hate and their envy have already perished, and they have no more share forever in anything that is done under the sun.
\end{quoting}

She told me it was from the Bible, but she couldn't remember which book. It didn't sound very biblical to me.

Abercastle was scarcely half a mile further down the coast. We sat on a bench overlooking its rocky beach. We were waiting for George to appear in the car park. Then Paula began talking about Tolstoy's \textit{Confession}. She had first read it in the lower sixth; one of the girls had brought a copy into the dorm, and it had caused an unlikely sensation as it made the rounds. `I suppose we were a very serious bunch, but there wasn't much to do in the evenings,' she said with a smile. `You made your own entertainment.' It's a slim volume, especially by Russian standards; there are probably crisp packets with a higher word count in that part of the world. The author asks himself whether life can have any meaning given the total annihilation to which we are all heading -- whether, in fact, life can be bearable at all. He wrote it in response to a midlife crisis, but it dovetails with teenage angst rather well. `It ends with a description of a dream he had had recently,' she said. `He's conscious of a void below him and a light above him, and he's floating between the two, but he doesn't understand what's keeping him afloat. He panics, and is actually about to drop into the darkness, until a voice tells him that, so long as he keeps his eys on the light, no harm can come to him, and so that's what he does. That image has stayed with me ever since, and it definitely describes where I am right now.'

`You've shown a lot of yourself to me today,' I said. `I know there's a school of thought that says you should never show yourself as you truly are, that if you show yourself naked to your friend, he cannot help losing respect for you. But I want you to know that I reject that school of thought completely.'

`I'm afraid, old thing, that I have absolutely no desire to see you naked.'

Then we both dissolved into laughter.

\section{}

It finally happened on 10 April 1994, a Sunday, the night of the network premiere of \textit{Goodfellas} on Channel 4. I remember that because we watched it together -- Ellie, Paula and myself -- in the farmhouse living room. I was really excited; I had heard a lot about the film; I had been too young to see it when it came out in the cinema. Ellie opened a couple of bottles of wine -- Chablis, unfortunately -- I teased her that the occasion surely cried out for a robust Italian red. Paula speculated (with a broad smile) on the volume of bloodshed we were about to witness; but then, just after that scene where Billy Batts gets repeatedly perforated with a carving knife, she said that it was past her bedtime, and disappeared upstairs. Ellie moved off the armchair to sit next to me on the sofa. Then, some time later, she said it would be good if I were to visit her bedroom in a few minutes. I remember I switched off the TV just after that montage where they pull that guy out of the freezer: `When they found Carbone in the meat truck, he was frozen so stiff it took them two days to thaw him out for the autopsy.'

I worked up the courage and pushed open the door. Only a small bedside lamp was on, giving the room a cosy glow. She was lying in bed with the duvet pulled up to her chin, like a little girl waiting to be tucked in. I hovered for a few moments by the side of the bed.

`Come here,' she said, and pushed the covers to one side.

She was as naked as the day she was born. Her breasts were larger and whiter than I had pictured them. I realised that the familiar caramel of her face and arms must have come from her fondness for team games and middle-distance running; the rest of her was so pale. Her shoulders were thick and sinewy, as if they belonged to a boy in his late teens; that wasn't my thing, but neither did it put me off. She had a recent French bikini wax, as was fashionable in those days. The carpet matched the curtains.

I took my clothes off and lay down next to her.

`Can I kiss you?' I asked.

`Just a little bit.'

Her lips tasted of booze. She opened her eyes very wide when I entered her, but I think it was out of simple discomfort, and not any kind of ecstasy or epiphany. A couple of minutes into it, I got the feeling that, while she was trying her best to make me feel comfortable and to encourage me to continue, she wasn't having a good time; so I climaxed as quickly as I could. There was an attempt at cuddling afterwards, but I found myself lost for words, and I couldn't bear to spoil what for me was a sacred moment by talking about the weather or her last meal. So I told her I would leave her in peace, and got dressed. She kissed me on the cheek before I left.

I wandered down to the kitchen. I noticed a 375 ml bottle of vodka on one of the window sills, half-hidden behind a vase. There wasn't much left. I poured myself a shot and drank it, and then another. That was the last of it. Then I went to bed.

I didn't see \textit{Goodfellas} all the way through until I was thirty-five.

\section{}

After that night, we did it two or three times a week, depending on how often she was home. We largely followed the same script each time. Then around the beginning of July she informed that she was pregnant, and thus my services would no longer be required. (I'm sure she must have put it more gently than that, but that's how I remember the conversation unfolding.) I was overjoyed, of course, but a part of me was also crushed by the news. For all the strangeness, sleeping with her meant the world to me.

Ellie mentioned years later that her first trimester had been really horrible, but at the time I had no idea. In my defense, she kept me at arm's length throughout the pregnancy; I was lucky if the three of us ate together even once week. By September, Paula's decline had accelerated visibly; she was struggling to grip a pint glass properly, even with her good hand, and she was starting to slur her words. By Christmas, her walking had become unsteady. At a lunch with a dozen or so of her wider friend group, she remarked that she was now at a stage of life where difficulty walking was a sign of decrepitude, rather than a vigorous romantic life, and brought down the house.

\section{}

My son Paul was born on 23 April 1995. As a Welshman, I wished it was any other date. As an admirer of \textit{The Life of Henry the Fifth}, that magnificent play, I have learned to live with it:

\begin{quoting}
    All the water in Wye cannot wash Your Majesty's Welsh blood out of your body, I can tell you that. God bless it and preserve it as long as it pleases His Grace and His Majesty too.
\end{quoting}

Ellie gave birth at Withybush Hospital in Haverfordwest. Paula was in with her, not me. (I was in the carpark sipping a warm can of Dr Pepper.) She even cut the umbilical cord, so I'm told, although by that stage she must have needed some help from the midwife in terms of grip strength. He took Ellie's surname, as agreed, but I was allowed to choose his middle name, and so I gave him my own Christian name.

I remember him hiccoughing in Ellie's arms when I came to see them -- the sweetest music I have ever been blessed with hearing -- it was the very first evening of his little life. But raising infants comes more naturally to women than to men, and it became clear to me that, in the imitation of the Holy Family which Ellie was attempting to construct, I had no place or role. I no longer felt entirely welcome at the farmhouse. When I look back at how little I saw of my son in the first few months of his life, I feel both anger and shame.

But Paula was visibly dying by then. One soul had entered the world as another was just leaving, as if the universe had limited seating. By June, she had basically lost the use of both arms and both legs. By the end of July, she was beginning to have trouble breathing. Ellie, adamant that Paula would live her last days at the farmhouse, converted the living room into a kind of cottage hospice.

\section{}

She never truly lost the power of speech, however. I remember one time I came to see her, just after she had finished her lunch. She had given herself a nasty knock on the head the night before, falling out of bed, and although, given her bleak prognosis, it wasn't really anything worth getting too worked up about, Ellie had panicked and called an ambulance.

`I was lying there,' Paula explained, `and suddenly there was a woman in a bottle-green dress standing over me. She was beautiful. She had the most gorgeous chestnut-coloured hair. She knelt beside me and told me everything was going to be okay. I think I started crying, and she asked me why I was upset, and I told her that it made me sad to be dying when there was such a beautiful creature in the world. Then when I came round properly in the back of the ambulance, I realised that I had been talking to the paramedic -- a dark fella with dreadlocks.'

That was the last time I saw her. She died six days later. Her breathing became shallower and shallower, until she stopped breathing all together. She didn't suffocate or anything like that. She more or less died in her sleep. That's what I was told. Ellie asked me to stay away from the farmhouse at the end, when it was clear Paula only had a day or so left and the various delegations of friends and family arrived wanting to hold her hand. I wasn't allowed to go to the funeral either. I suppose my absence was necessary to maintain the fiction that two women could make a living soul without male help. I try not to judge Ellie's reasoning too harshly; were it not for Ellie's reasoning I would never even have met Paula, nor would I have my son. But I still feel angry about it, even now. She was my friend too.

\section{}

A few months after Paula died -- it must have been in October -- Ellie invited me to the farmhouse for an evening. It was the week that Paul turned six months, and she wanted us to celebrate it together. We watched \textit{The Little Mermaid} on VHS, although Paul was of course too young to take much of it in. Afterwards, we put him to bed; no trouble to anyone, that lad, after seven o'clock at night. Then Ellie asked me to stay and have a drink with her -- she had recently stopped breastfeeding -- and we ended up finishing three 50 cl bottles of brandy between us. Then the next thing I knew I woke up at four in the morning on the living room sofa, and shuffled off to spend the rest of the night in one of the guestrooms.

I woke up at half seven to the sound of one of Ellie's staffers making breakfast. I started to get out of bed, and was wishing I had never been born before my feet touched the carpet. I went to the bathroom. (The room had an en suite.) I felt a violent urge to be sick -- I was literally the most hungover I have ever been in my life -- but nothing would come out. So I stood under the shower for twenty minutes, and then, putting on a dressing gown, made my way to the kitchen. Paul was there in his high chair -- he had just learned to sit up -- charming the socks off of anyone who would give him so much as a glance. Ellie pulled out the chair next to her.

`Have a seat,' she said. `You look like you're about to pass out.'

I poured myself a mug of tea and chewed half-heartedly on a lightly-buttered slice of cold toast. Ellie showed me how to feed Paul scrambled eggs. A welcome shaft of light flooded the room. Staffers tidied away most of the breakfast things, then disappeared themselves. Ellie was telling me about an ongoing commercial acquisition she was tearing her hair out over. Then, for no earthly reason, I put my arm around her waist.

She removed it immediately.

`I'm sorry,' I said.

`It's all right. You must be famished.' She paused. `If you need someone to keep you company, I'm sure George would be able to sort something out.'

`No doubt that man could find a knocking shop blindfolded in the Wilderness of Zin,' I said, which made her smile.

Then I made my excuses, kissed Paul on the top of his head, and went back to my cottage.

I suppose I should tell you at this point how I kept myself pure for her, how I tended my love for her like a sacred and immaculate flame. But George's skill as a procurer exceeded all expectations, and I was indeed famished. Over an eighteen-month period, I enjoyed the company of thirty-five women -- a new girl every fortnight, or thereabouts -- I have the names and dates all written down. I read somewhere that God forgives hungry men. I hope that's true.

\section{}

Of course, the first time I was with a call girl I was very nervous. But it was an older woman -- she said she was thirty-eight -- and she was very kind to me. After the first few times, I developed the whole process into a kind of ritual, and that extinguished most of my anxiety.

I would meet the lady in question on a Saturday lunchtime -- at her place, if it was suitable, otherwise a hotel. I remember one girl; I had to drive up into the Preselis to see her; she lived in a tiny terraced house in Rosebush. She was American. Although I had watched all the usual films and TV shows, I had never actually met an American in the flesh before then. As silly as it might sound, I was actually a little bit star-struck by her. She told me her name was Cassidy. When I phoned her at eleven in the morning to confirm, she said that she'd totally forgotten about my booking. But that was fine; I could come over any time I liked. Then when she answered the door, she was visibly distressed that I had got her out of bed first thing in the afternoon, and I realised that I'd just found my soulmate.

She was five foot one with freckles and bleached-blond hair cut into a bob. Mid-twenties. Dark blue eyes. Before we began, she said that she had been at a swingers' party the night before. She was self-conscious about the bruises on her legs; she had misplaced her last good pair of stockings. I should be extra gentle. Then, when we were doing it, she hugged me and told me I could go harder if I liked.

Afterwards, she wanted me to hold her -- a rarity among ladies of her profession, at least in my experience. Of course, I was only too happy to oblige, and, with her head in the crook of my shouder, she told me the unlikely tale of how she had made her way from Pittsburgh, Pennsylvania to a glorified hamlet in rural Wales. She had strung up fairy lights around the headboard, and beside us there was a large window with a good view of the rugged landscapes and the several hill farms into which it had been plotted and pieced. Then she asked me if I could drive her to the Tesco's in Haverfordwest; she didn't have a car. It was on my way, and of course I'd be glad to help in any case. It was early spring, still cold but quite sunny. It felt like the earth itself was waking up after a pleasant dream.

I remember the two of us kissing in the car park once we had arrived. That was the last time I saw her. (She said she could make her own way back.) She told me she wanted to see me again, but, at the time, I felt like another emotional entanglement was the last thing I needed.

Of course, not all my experiences with call girls were as wholesome as that. I don't mean to play down the evils that are linked, probably inextricably, to prostitution. But, if I'm honest with myself, it's a period of my life which I look back on with a certain fondness and with few regrets. Naturally, I had a couple of disappointing encounters, but I never had what people call a bad experience. I have no recollection of causing anyone else a bad experience; I dearly hope my recollections are true to life in that respect. Later, I experimented with having two, sometimes three, girls at once. Then, having discovered the limits of what was possible, I lost interest in the whole thing.

\section{}

You already know, of course, that I kept living in the cottage which Ellie gave me. I could have moved away if I had wanted to; it wouldn't have have stopped my annuity or anything like that. But it's an arrangement which has allowed me to watch my son grow up, which has been the joy of my heart. He's eleven now. I see him more or less every day. He has his own room in my cottage; he stays with me most weekends. I let him stay up far too late. We share a weakness for obnoxiously violent films; I'm sure I'm too permissive in that regard, but I regret nothing. I see myself in him, not so much in the big things as in little eccentricities I had assumed were unique to me; we both like to read, for instance, with our eyeballs three or four inches from the page. He's ahead of the rest of his class with his maths, which naturally I attribute to my tutoring. He asked me recently if he could use my surname when he goes to his next school. I told him that that would make me very happy indeed, but he would have to get his mother's permission as well.

I think about Paula all the time. A cynical person might think I would have been glad to have had a rival taken out of the picture. But, truly, I miss her a great deal. Like a bar stool missing one of its rubber feet, nothing in my life is quite right without her being there to talk to.

I went through a kind of religious awakening about a year and a half after she died. It began on a February afternoon, in Durham. I was visiting a friend who was finishing his PhD at the university there. I was sitting alone, upstairs in a coffee shop. I remember it was desperately cold outside. I have always felt self-conscious about not finishing my degree, and I thought that Wittgenstein's \textit{Tractatus} might be a bracing reintroduction to mathematical logic. It wasn't. But then, leafing through the latter part of the book merely to pass the time until the aforementioned friend arrived, I stumbled on a passage that knocked the wind out of me:

\begin{quoting}
    Death is not an event in life: no one lives to experience death. If we take eternity to mean not infinite temporal duration but timelessness, then eternal life belongs to those who live in the present. Our life has no end in the same the way in which our visual field has no limits.

    Not only is there no guarantee of the temporal immortality human soul, that is to say, of its perpetual survival after death; but, in any case, this supposition completely fails to accomplish the purpose for which it has always been intended. Or is some riddle solved by my surviving forever? Is this version of eternal life not itself as much of a riddle as our present life? The solution of the riddle of life in space and time lies outside space and time.
\end{quoting}

That was the passage which, as I said, knocked the stuffing out of me. It was the first thing, other than the passage of time, which attenuated that grief which had swallowed my heart. But another passage, a couple of paragraphs further on, had more of an impact on my subsequent life:

\begin{quoting}
    Scepticism is not irrefutable, but obviously nonsensical, when it tries to raise doubts where no questions can be asked. For doubt can exist only where a question exists, a question only where an answer exists, and an answer only where something can be said.

    We feel that even when all possible scientific questions have been answered, the real problems of life remain completely untouched. Of course there are then no questions left, and this itself is the answer. The solution of the problem of life is seen in the vanishing of the problem. (Is this not why those who have found, after a long period of doubt, that the sense of life suddenly became clear to them, have then been unable to say what constituted that sense?)
\end{quoting}

I reread Mark's Gospel, this time with a heart of flesh. I was startled by the immediacy and economy of Mark's storytelling. Those precious details: the pillow under Christ's head as he slept through the storm; the blackest-of-black comedy of the young disciple fleeing from the Garden of Gethsemane wearing nothing but a pair of sandals. I retraced the paths I used to walk with Paula, not in hope of recapturing anything but because those places are intensely beautiful in and of themselves; and yet the idea of heaven and earth coming together is more plausible in this landscape, I think, than any other. God is absent, at least it feels that way, but that absence itself is a kind of presence; like the rainbows so common among these rainy, sun-bleached headlands, God is both conspicuously immanent and permanently out of reach.

Ellie's drinking, which in the beginning had been an endearing peccadillo, grew into something larger and uglier over the years. She was predominantly, I have to say, a good-natured drunk, but a darker side would show itself from time to time. She could be unreasonable, and she knew me well enough that she could crush my self-worth in a few words. In any case, I just tried to stay out of her way in the evenings. She became withdrawn, not in a way that would be obvious to an outsider looking in, but we went from a relationship in which I felt like I could come to her with any crisis to one in which I felt uneasy if the two of us were ever alone. A few times local restaurants phoned me up mid-afternoon; she had outstayed her welcome, and was in no state to make her own way home. I tried to talk to her about it at least a dozen times, but either my courage failed me or she would brush me off before we could get anywhere. After a few years of all that, I came to accept that I couldn't save her from whatever it was that had taken hold of her. I suppose there are spots in her mind too tender ever to be touched, and she would rather drink herself to death than have a meaningful conversation about it.

\begin{quoting}
    \begin{verse}
        This is the deadly spite that angers me:\\*
        My wife can speak no English, I no Welsh.
    \end{verse}
\end{quoting}

Except of course that we never married, and I only started learning Welsh in secondary school.

As unlikely as it might sound, none of the above ever really bled into her professional life. On the contrary, given the landslide New Labour victory of 1997 and the Blair Boom which followed, her success as an investor reflected the general exuberance of the times. I can't say whether a rising tide truly lifts all boats -- I don't have an evangelical commitment to any one economic theory or other -- but I did see her little flotilla rise quite handsomely. She became a minor celebrity, the sort of businesswoman whose faith and morals schoolgirls are instructed to emulate. What Paula would have made of all that I really don't know.

I remember one evening last summer. I was driving back from Newgale Beach; I had just done the walk to Solva and back. I think it was about six o'clock. Being late June, it was still broad daylight. I was listening to Radio 3, which is odd, because I only ever listen to talk radio in the car; I must have hit one of the preset buttons inadvertently. I don't really have much of an ear for music, but, before I could change station, they started playing something just astonishingly exquisite; the only way I can describe the sensation is that it was like when a dentist stabs you in the gum, except it was pure ecstasy, not pain. Thank goodness they played the full ten minutes. I know now that the piece is called \textit{Spem in alium}, by Thomas Tallis. The words are in Latin, taken from one of the moth-eaten missals of the Sarum Rite, but I understand that they translate to something like this:

\begin{quoting}
    \begin{verse}
        I have never hoped in another\\*
        Than you, Lord God of Israel,\\
        Who shows both anger and mercy,\\
        And who forgives, through suffering,\\*
        All the sins of men.
    \end{verse}
\end{quoting}

The music made me feel -- still makes me feel -- as if my eyes have been truly opened for the first time, as if I have discovered a whole new world, and it would be the adventure of a lifetime just to explore one small piece of it. It speaks to me of heaven invading earth -- if I close my eyes, I see angels storming the beaches and beams of golden light shattering battleships and tanks -- but also of that crazy, lovesick, white-knuckle desire to see God in the face, which I do feel -- right to my bones.

If I'm on my own in the cottage, sometimes I'll sit at my desk and play a bit of Thomas Tallis for company. I'm very conscious of how much I have to be grateful for, and how little I can reasonably complain. I have a healthy son, a ruby which several kings and emperors dreamed of but never achieved. I'll never have to worry about money; it's only when I hear about the various kinds of hell most people have to put themselves through just to keep body and soul together that I begin to realise what a blessing that is. But if some genie were to whisper in my ear that he could give me anything in the world, I would always choose her.
