\renewcommand{\thesection}{\Roman{section}} 

\proseepigraph{...and for my thirst they gave me vinegar to drink.}\blindfootnote{The title is taken from Psalm 69 (LXX: 68), as is the epigraph. The author dedicates this story to his father.}

\section{}

How did I meet her? I was living in Pembrokeshire at the time -- that's where I'm from -- and I was working as a cashier at a tourist spot. It was a dairy farm once upon a time; then in 1986 the herd manager decided to turn it into a kind of rustic theme park, adding donkey rides, a fairground, candyfloss, even a gift shop. It was a shrewd decision. She -- the proprietress -- had just come into her inheritance. (There were stories about her father having fallen off of his yacht, but I believe he actually died peacefully in a hotel room in Picadilly.) The old farm happened to be one of her family's holdings. She decided to pay us a visit, inspect the troops, or something like that; that was what we were told. She came down to our section when most of us were on a break, and, because this was the early 1990s and it was a nice day, everyone was smoking. Except I wasn't: I've ever been precious about my health, or anything; I actually like the smell; I've just never been able to tolerate the sensation of smoke in my throat. She did try to be friendly -- she talked to a couple of us -- but she couldn't stop herself from making a face whenever a puff of smoke drifted towards her nostrils. As she left, she touched me lightly on my right shoulder. I know now that that was the signal for her staff: this is the one I want.

When I turned up for work a couple of days later, my boss asked me if I could work late. I told him yes, and he explained that I had a special task; upper management were canvassing frontline workers. Best behaviour, and all that. A car arrived at five o'clock to pick me up. The driver's name was George; he told me he had worked the proprietress' family all his life, in one role or another. He drove me to an old stonebuilt farmhouse -- from the front, it looked like it had six or seven bedrooms -- about fifteen minutes out of Milford Haven. It was a good spot: half a mile from the Coastal Path, but still with a good view of the sea. George showed me into the front room, and there was the proprietress to meet me with Welsh cakes and a pot of coffee.

We talked about nothing for maybe twenty minutes. She told me to call her Ellie. She asked me how I was treated at the farm. I couldn't complain. Then she pointed out that I'd barely touched my coffee since I'd sat down.

`To tell you the truth, I don't really drink coffee. I'm not a Mormon or anything. I just don't like the taste.'

She asked me if I would prefer a proper drink. I said I would, so she told me to follow her. She led me upstairs, into what looked to be her own room.

`Take a seat,' she said. There was nowhere to sit but the bed. `I'm living out of a suitcase at the moment, but I always make sure the fundamentals are taken care of.' She gestured to one corner of the room, in which there was a kind of impromptu bar. `What can I get you?'

`Oh. John Smith's please, if you've got it,' I said, trying to be manly. It was a warm evening. Had she not been there to impress, I would have poured myself a gin and tonic.

`I do have it.' She handed me a can and a pint glass. Then she poured out a bottle of Old Peculier, and, having admired her work for a few moments, took a small sip.

`Forgive me for saying so,' I said, `but you don't often see a woman enjoying that kind of beer.

She smiled at me. She had an exceptionally warm smile, almost comical, as if her mouth was too big for her face. `My father was a recovering alcoholic,' she said. `When I was little, my mum used to ship me off to stay with him for a few weeks. He had a farm in Normandy. He taught me to drive a tractor. And I used to drink his non-alcholic beer. I was eight years old, and I loved the taste of it straight away.'

She asked me to tell her about myself. I told her there wasn't much to tell. She told me to tell her anyway. So I told her my story: how I had been studying physics at Bristol, how I had got my girlfriend pregnant, how I had dropped out and found a job, how she had had an abortion at twenty-three weeks.

I noticed her widening her eyes in sympathy. The she pulled a book from the shelf nearest the bed and sat down next to me. She opened the book at a specific page and gave it to me; it was a poem.

`I can't read French,' I explained. `If it was Welsh, I might have half a chance, but --'

`That's all right,' she said. `I can translate for you.' She put her finger on the first line: `\textit{Il pleure dans mon coeur} -- It rains in my heart.' I forget the rest.

Once she had finished reading -- I think it was only a dozen or so lines -- she closed the book and put it back. Then she kissed me. When I opened my eyes, I noticed that she was already watching me, and it seemed as if she had been watching my reaction the whole time, like I was an amuse-bouche, one of several, and she was trying to decide which one she liked best. Her eyes were a very vibrant blue, like the waters off the Costa del Sol in the glossy package holiday brochures they had back then. She was sorry, she said, but she had a meeting in London the following morning; she had to leave shortly, and she hadn't yet packed. George would drive me home.

`But one more thing,' she said as I began to get up, and kissed me again.

\section{}

George picked me up after work again the following week, and brought me back to Ellie and her farmhouse. I remember her hugging me as soon as I walked in. We sat on the floor of her room and talked. We drank a bit, but not enough to make me tipsy, and we each said some more about our lives up until then. She tried to teach me how to play canasta. Then at some point she put her head on my shoulder, and we started kissing.

I could feel my soul starting to wrap itself around the image it was forming of her. If that sounds a bit much, bear in mind that I had only been with one woman at that time, and I had only just turned twenty. I unbuttoned my shirt and, taking her right hand, held it against my chest, so that she could feel my heart beating under my skin. She, in turn, held my hand against her blouse, although, as she explained, she had never had much success finding her own heartbeat. I told her that I couldn't quite find it either, but it was a very nice breast all the same. Then she kissed me and made the same excuse as last time -- something about an early start the following morning -- and George drove me home as before.

\section{}

I thought I must have said something to upset Ellie, because I didn't hear from her again for almost two months. But then George appeared in the breakroom one morning just before lunch; I was to go with him to the farmhouse right away. Ellie was standing in the dining room when I arrived. She asked me to take a seat.

She did not seem to be quite herself. She was wearing a skirt and a blazer, but had taken off her shoes, so that only her tights came between her toes and the slate tiles beneath them. She opened her mouth to say something, stopped, then tried again.

`For reasons which I am not willing to divulge at this time,' she said, `it is essential for me have a child within the next eighteen months, and I would like to know whether you would be willing to assist me in achieving this goal.'

`Gosh,' I said. `Forgive me if I've got the wrong end of the stick, but you want the two of us to have a baby together?'

She nodded.

Seeing the agony written over her face, I continued: `I imagine someone's told you by now how babies are made?'

The tension that had been ratcheting up in her chest seemed to ease off by half a click. She said she wanted to do things as quietly and straightforwardly as possible, and natural methods were more conducive to that than the alternatives.

Did she want anything more from me than a means of conception?

`I would like my child to know his or her biological father,' she said. `But one thing you must understand if we are going to proceed any further: I am not looking for any kind of husband or boyfriend, and I never will be.'

She laid out the terms of the arrangement in more detail. Too much detail, to be honest, but she was talking excitedly like she just had to talk, so I reasoned it was better to let her offload whatever was on her mind. My name would be kept off the birth certificate. I would receive an annuity: nothing spectacular, but several times more than the amount required just to keep body and soul together. And I was welcome to stay at the farmhouse -- there was an annexe with several empty bedrooms -- whenever, and for as long as, I liked.

I spoke as soon as a gap opened up in Ellie's monologue: `Well,' I said, `today's your lucky day.'

\section{}

I first met Paula in a pub in Rosyth. It was late September, an suprisingly balmy afternoon. Ellie had suggested that, before we jumped into the deep end of co-parenthood, we ought to get to know each other first. You get to know people through the things they love, and it turned out that the thing she loved more than anything else in the world was field hockey; she was the centre forward for an ill-fated London team, which had just lost 6-1 in a friendly to Dumfermline Ladies. So there I was at the bar, attempting to drown my boredom with players' husbands and sweethearts. I ordered a pint of Tennent's; the barmaid handed it to me, and, as I grasped the glass, it slipped through my fingers, soaking the bar mat and dripping onto the floor. I felt a hand squeeze my right shoulder.

`Don't worry, old thing. Same thing happened to me the other day. Let me buy you another. I'm Paula, by the way.'

We sat down at a table next to a large window, looking out over the Firth of Forth. The water was calm. It was one of those placid September late afternoons.

`What did you think of the hockey, then?' she asked me.

`Ball games aren't really my thing,' I said, `but there are worse ways of spending a Saturday than watching twenty-two girls in short skirts reenact the Battle of Flodden.'

She smiled indulgently at that remark, then, looking out of the window for a moment, replied, `Well spoken, sir. Except that the English won Flodden, if I remember correctly.'

Either it was my imagination or she gave me a little wink. Then we talked for a bit about the team's chances for the coming season. There was little room for optimism. Ellie was our best player, and it was likely that she would have to miss a lot of the upcoming training sessions, maybe even some matches, given the responsibilities she had acquired with her inheritance. Paula herself had had to step back from the team; a recent injury had rendered her, in her own words, `a pretty atrocious goalie'.

Something I asked caused her to launch into a lengthy digression concerning a recent change the offside rule, and, instead of listening properly, I began to study her face. She was noticeably sturdier than Ellie, but she was quite beautiful in her own way. She had a clever, almost feline-looking countenance, and sparkling green eyes enhanced by a few touches of mascara. Just as I was beginning to wonder about her age -- thirty, perhaps? -- she got up, and, touching me on the shoulder, said in half a whisper, `I'm glad you're just as handsome as Ellie said you were. Best of luck to you.'
