\renewcommand{\thesection}{\Roman{section}} 

\proseepigraph{...and for my thirst they gave me vinegar to drink.}\blindfootnote{The title is taken from Psalm 69 (LXX: 68), as is the epigraph. The author dedicates this story to his father.}

\section{}

How did I meet her? I was living in Pembrokeshire at the time -- that's where I'm from -- and I was working as a cashier at a tourist spot. It was a dairy farm once upon a time; then in 1986 the herd manager decided to turn it into a kind of rustic theme park, adding donkey rides, a fairground, candyfloss, even a gift shop. It was a shrewd decision. She -- the proprietress -- had just come into her inheritance. (There were stories about her father having fallen off of his yacht, but I believe he actually died peacefully in a hotel room in Picadilly.) The old farm happened to be one of her family's holdings. She decided to pay us a visit, inspect the troops, or something like that; that was what we were told. She came down to our section when most of us were on a break, and, because this was the early 1990s and it was a nice day, everyone was smoking. Except I wasn't: I've ever been precious about my health, or anything; I actually like the smell; I've just never been able to tolerate the sensation of smoke in my throat. She did try to be friendly -- she talked to a couple of us -- but she couldn't stop herself from making a face whenever a puff of smoke drifted towards her nostrils. As she left, she touched me lightly on my right shoulder. I know now that that was the signal for her staff: this is the one I want.

When I turned up for work a couple of days later, my boss asked me if I could work late. I told him yes, and he explained that I had a special task; upper management were canvassing frontline workers. Best behaviour, and all that. A car arrived at five o'clock to pick me up. The driver's name was George; he told me he had worked the proprietress' family all his life, in one role or another. He drove me to an old stonebuilt farmhouse -- from the front, it looked like it had six or seven bedrooms -- about fifteen minutes out of Milford Haven. It was a good spot: half a mile from the Coastal Path, but still with a good view of the sea. George showed me into the front room, and there was the proprietress to meet me with Welsh cakes and a pot of coffee.

We talked about nothing for maybe twenty minutes. She told me to call her Ellie. She asked me how I was treated at the farm. I couldn't complain. Then she pointed out that I'd barely touched my coffee since I'd sat down.

`To tell you the truth, I don't really drink coffee. I'm not a Mormon or anything. I just don't like the taste.'

She asked me if I would prefer a proper drink. I said I would, so she told me to follow her. She led me upstairs, into what looked to be her own room.

`Take a seat,' she said. There was nowhere to sit but the bed. `I'm living out of a suitcase at the moment, but I always make sure the fundamentals are taken care of.' She gestured to one corner of the room, in which there was a kind of impromptu bar. `What can I get you?'

`Oh. John Smith's please, if you've got it,' I said, trying to be manly. It was a warm evening. Had she not been there to impress, I would have poured myself a gin and tonic.

`I do have it.' She handed me a can and a pint glass. Then she poured out a bottle of Old Peculier, and, having admired her work for a few moments, took a small sip.

`Forgive me for saying so,' I said, `but you don't often see a woman enjoying that kind of beer.'

She smiled at me. She had an exceptionally warm smile, almost comical, as if her mouth was too big for her face. `My father was a recovering alcoholic,' she said. `When I was little, my mum used to ship me off to stay with him for a few weeks. He had a farm in Normandy. He taught me to drive a tractor. And I used to drink his non-alcholic beer. I was eight years old, and I loved the taste of it straight away.'

She asked me to tell her about myself. I told her there wasn't much to tell. She told me to tell her anyway. So I told her my story: how I had been studying physics at Bristol, how I had got my girlfriend pregnant, how I had dropped out and found a job, how she had had an abortion at twenty-three weeks.

I noticed her widening her eyes in sympathy. The she pulled a book from the shelf nearest the bed and sat down next to me. She opened the book at a specific page and gave it to me; it was a poem.

`I can't read French,' I explained. `If it was Welsh, I might have half a chance, but --'

`That's all right,' she said. `I can translate for you.' She put her finger on the first line: `\textit{Il pleure dans mon coeur} -- It rains in my heart.' I forget the rest.

Once she had finished reading -- I think it was only a dozen or so lines -- she closed the book and put it back. Then she kissed me. When I opened my eyes, I noticed that she was already watching me, and it seemed as if she had been watching my reaction the whole time, like I was an amuse-bouche, one of several, and she was trying to decide which one she liked best. Her eyes were a very vibrant blue, like the waters off the Costa del Sol in the glossy package holiday brochures they had back then. She was sorry, she said, but she had a meeting in London the following morning; she had to leave shortly, and she hadn't yet packed. George would drive me home.

`But one more thing,' she said as I began to get up, and kissed me again.

\section{}

George picked me up after work again the following week, and brought me back to Ellie and her farmhouse. I remember her hugging me as soon as I walked in. We sat on the floor of her room and talked. We drank a bit, but not enough to make me tipsy, and we each said some more about our lives up until then. She tried to teach me how to play canasta. Then at some point she put her head on my shoulder, and we started kissing.

I could feel my soul starting to wrap itself around the image it was forming of her. If that sounds a bit much, bear in mind that I had only been with one woman at that time, and I had only just turned twenty. I unbuttoned my shirt and, taking her right hand, held it against my chest, so that she could feel my heart beating under my skin. She, in turn, held my hand against her blouse, although, as she explained, she had never had much success finding her own heartbeat. I told her that I couldn't quite find it either, but it was a very nice breast all the same. Then she kissed me and made the same excuse as last time -- something about an early start the following morning -- and George drove me home as before.

\section{}

I thought I must have said something to upset Ellie, because I didn't hear from her again for almost two months. But then George appeared in the breakroom one morning just before lunch; I was to go with him to the farmhouse right away. Ellie was standing in the dining room when I arrived. She asked me to take a seat.

She did not seem to be quite herself. She was wearing a skirt and a blazer, but had taken off her shoes, so that only her tights came between her toes and the slate tiles beneath them. She opened her mouth to say something, stopped, then tried again.

`For reasons which I am not willing to divulge at this time,' she said, `it is essential for me have a child within the next eighteen months, and I would like to know whether you would be willing to assist me in achieving this goal.'

`Gosh,' I said. `Forgive me if I've got the wrong end of the stick, but you want the two of us to have a baby together?'

She nodded.

Seeing the agony written over her face, I continued: `I imagine someone's told you by now how babies are made?'

The tension that had been ratcheting up in her chest seemed to ease off by half a click. She said she wanted to do things as quietly and straightforwardly as possible, and natural methods were more conducive to that than the alternatives.

Did she want anything more from me than a means of conception?

`I would like my child to know his or her biological father,' she said. `But one thing you must understand if we are going to proceed any further: I am not looking for any kind of husband or boyfriend, and I never will be.'

She laid out the terms of the arrangement in more detail. Too much detail, to be honest, but she was talking excitedly like she just had to talk, so I reasoned it was better to let her offload whatever was on her mind. My name would be kept off the birth certificate. I would receive an annuity: nothing spectacular, but several times more than the amount required just to keep body and soul together. And I was welcome to stay at the farmhouse -- there was an annexe with several empty bedrooms -- whenever, and for as long as, I liked.

I spoke as soon as a gap opened up in Ellie's monologue: `Well,' I said, `today's your lucky day.'

\section{}

I first met Paula in a pub in Rosyth. It was late September, an suprisingly balmy afternoon. Ellie had suggested that, before we jumped into the deep end of co-parenthood, we ought to get to know each other first. You get to know people through the things they love, and it turned out that the thing she loved more than anything else in the world was field hockey; she was the centre forward for an ill-fated London team, which had just lost 6-1 in a friendly to Dumfermline Ladies. So there I was at the bar, attempting to drown my boredom with the other players' husbands and sweethearts. I ordered a pint of Tennent's; the barmaid handed it to me, and, as I grasped the glass, it slipped through my fingers, soaking the bar mat and dripping onto the floor. I felt a hand squeeze my right shoulder.

`Don't worry, old thing. Same thing happened to me the other day. Let me buy you another. I'm Paula, by the way.'

We sat down at a table next to a large window, looking out over the Firth of Forth. The water was calm. It was one of those placid September late afternoons.

`What did you think of the hockey, then?' she asked me.

`Ball games aren't really my thing,' I said, `but there are worse ways of spending a Saturday than watching twenty-two girls in short skirts reenact the Battle of Flodden.'

She smiled indulgently at that remark, then, looking out of the window for a moment, replied, `Well spoken, sir. Except that the English won Flodden, if I remember correctly.'

Either it was my imagination or she gave me a little wink. Then we talked for a bit about the team's chances for the coming season. There was little room for optimism. Ellie was our best player, and it was likely that she would have to miss a lot of the upcoming training sessions, maybe even some matches, given the responsibilities she had acquired with her inheritance. Paula herself had had to step back from the team; a recent injury had rendered her, in her own words, `a pretty atrocious goalie'.

Something I asked caused her to launch into a lengthy digression concerning a recent change to the offside rule, and, instead of listening properly, I began to study her face. She was noticeably sturdier than Ellie, but she was quite beautiful in her own way. She had a clever, almost feline-looking countenance, and sparkling green eyes enhanced by a few touches of mascara. Just as I was beginning to wonder about her age -- thirty, perhaps? -- she got up, and, touching me on the shoulder, said in half a whisper, `I'm glad you're just as handsome as Ellie said you were. Best of luck to you.'

\section{}

That sounded like a permanent valediction. In fact, Paula came to stay at the farmhouse quite frequently that autumn, as did I. That was how she came to tell me her story. She and Ellie had met in their teens, at an expensive boarding school which had reluctantly gone coeducational. Both were keen to point out that cost was no guarantee of quality. Both had trouble fitting in. Even though Paula was two years older, an eternity at that age, they had become close friends. When Paula failed to get into Oxford -- `Balliol, modern languages,' she confided with a mixture of satisfaction and self-reproach -- she decided to change tack all together. She trained as a marine engineer -- she explained that she had long been fascinated with shafting -- and thereafter had been in the service of several cruise lines. She worked in the Mediterranean mostly. The trips were always four months long, but the spacing between them could be quite irregular -- sometimes a few weeks, sometimes almost a whole year -- depending on the needs of the company, the training courses necessary to maintain her certification, and how she felt. She had had some catastrophic falling out with her biological family; I never got to the bottom of it, but the effect was that, since her job meant that she was out of the country for big chunks of each year, and since she couldn't stay with her parents, she spent her leave living with Ellie more often than not. At the point that I became acquainted with her, she had had a permanent room at the farmhouse for some time.

They lived like a pair of wayward fourteen-year-olds each time they got together: staying up until four in the morning talking and drinking strawberry schnapps; watching the first four Freddy Krueger films in one go; smoking out of bedroom windows; subsisting entirely on jaffa cakes and sliced ham. They loved to tell each other the most appalling stories. One of Ellie's sticks in my mind; she claimed it had come from another of the girls she went to school with, who had recently qualified as a doctor. An ambitious young solicitor went to a clinic with a particularly unpleasant case of vaginitis. After several excruciating minutes in the stirrups, Ellie's acquaintance managed to pull a used prophylactic from deep within her. Apparently, it had been there at least a month. The patient's poor fianc\'e, meanwhile, was sitting in the waiting area as all this was going on. He was last seen in the car park, getting the hairdryer treatment from his beloved.

They were very fond of practical jokes too: a tablespoon of chili flakes in one's breakfast cereal, shampoo bottles topped off with mayonnaise. No marks for originality, but the execution was expertly done. Paula had an inflatable animal -- I forget whether it was Godzilla or a crocodile -- which, if you really blew it up, could get as big as a grown man. It used to turn up in all sorts of places: kitchen cupboards, behind shower curtains, in bed beside you when you woke up. Nor were innocent bystanders such as myself entirely safe from their mischief. For several months, I could not walk into my bedroom in the annexe without feeling that someone was about to jump out at me.

\section{}

So Ellie and Paula got to know me, and I them. Then one Wednesday afternoon -- it was about a fortnight before Christmas -- Paula and myself were alone in the house, because Ellie had had to go to London for a few days to attend to some sort of crisis in one of the family businesses. I was reading by the fire in the main house, a Jilly Cooper novel of all things. It was just starting to get dark, and I went to the bathroom, which caused me to pass by Paula's room. The door was open; she was sitting on the bed with her eyes closed; the colour had gone from her face. I rapped the door as gently as I could manage.

`Is everything okay?' I said. `Sorry, but I couldn't help noticing that you seem to be a bit under the weather.'

`It's fine,' she said, smoothing down her teeshirt. `I was praying.'

`Gosh. Forgive me, but you never struck me as the type.'

`What exactly are you insinuating?'

We both laughed, but the tension was still there.

Then she said, `I'm dying. I was asking if I might have a bit more time.'

`Dear God. Are you in pain?'

`No, no, not at all, sweetheart. It's okay. Ellie already knows; she's known for a while. Everything that can be sorted out has already been sorted out. You don't need to worry.'

`Am I allowed to ask what's wrong with you?'

`I have a neurodegenerative disease. Do you know what that is?'

I told her I did, but that she might have to bear with me if I misunderstood any of the specifics of her condition. Then she told me a bit more about what she had been through. She had her first symptom about a year ago; she was washing her car one winter afternoon, and her right hand had gone numb. At the time, she thought it must just have been the cold; she regained feeling after a couple of hours. But as the weeks passed her grip in that hand became weaker and weaker. Then the weakness spread into her arm.

`You might have noticed,' she said, `that I hardly use my right hand for anything, but I'm actually right-handed.' She got a firm diagnosis six months later. `No one can tell me when I'm going to die. I might last another year. I might last five.'

`Knowing my luck,' I said, `you'll be like that Stephen Hawking, and I'll be wheeling you around for the next thirty years.'

She laughed, thank goodness. We both did.

`Would it help if I stayed with you for a bit,' I said, `or would you prefer to be left alone?'

`No, I'd like you to stay.'

So I sat down next to her on the bed, and we prayed together in silence for about half an hour. I didn't have any specific religious faith at the time, and neither did she. I don't think either of us knew whether we were praying to Jupiter, Jehovah or Quetzalcoatl, but it didn't seem to matter. It just felt like the right thing to do in the circumstances.

\section{}

After Christmas, Ellie moved me out of the annexe and into a cottage a mile and a half away, on the edge of the estate. It was scandalous, Ellie liked to tease me, to keep a young man such as myself under the same roof as two respectable young ladies. I was still a part of their lives. I used to go round to the farmhouse on a weekend for a drink and a catchup. I had a thing about drinking negronis at the time, and Ellie had this fantastic bottle of red vermouth. (I actually couldn't tell you whether it was good quality, but the label was very ornate.) It was always a good time. Both of us knew about Paula's illness, and each of us knew that the other knew, but we would never, and never did, discuss it. But Ellie asked me if I could check on Paula, and perhaps have lunch with her, on those days when business called Ellie back to London, or wherever.

I grew closer to Paula during those lunches. She would walk down to my cottage; she insisted on making the most of her legs while they would still do as they were told. My skills as a cook are, to put it charitably, limited, but she seemed happy enough with sandwiches or boiled eggs. Once the weather became a bit more civilised, we used to venture out to the Coastal Path -- whenever I try to imagine the landscape of heaven, I always end up picturing Pembrokeshire in late spring -- but in those first few weeks we used to sit in my cosy front room and talk about whatever was on my mind. After the first couple of times, she would bring a musty old copy of the New World Translation along with her; she had been a Witness in a former life, she explained, or at least her parents were. She would read a few verses from one of the gospels out loud, and then say whatever it was that had touched her heart. Almost always, we would end up talking about something completely different, but it was helpful to have somewhere to start.

Her knowledge of the Bible was actually sort of shaky -- not that mine was any better at that time. In hindsight, I can see that a lot of the time she was just regurgitating undigested lumps of \textit{Watchtower} gobbledegook. Then every so often she would come out with something startlingly insightful, the sort of thing which would make me feel self-conscious that I would never have the spiritual horsepower to come up with anything like that myself. For example: we were looking at Matthew's version of the Sermon on the Mount, and when we got to the bit where Jesus tells the audience that they are the salt of the earth, she explained that this meant that human beings, having a sophisticated and self-reflective consciousness, are able to perceive the world more fully than any other animal, and thus give the world a flavour for which no other creature could substitute. I didn't have anything clever to say in response to that.

\section{}

Like I said, as the weather got warmer we tended to venture out more and more onto the Coastal Path, and have our conversations there. Nor did we limit ourselves to what was within walking distance of the estate. This was where it became really useful to have George on hand to drive us; he would drop us off in one of those tiny whitewashed seaside villages, and then pick us up a few miles round the coast. (No mobile phones in those days, but he seemed quite content to wait for us in the nearest pub with a newspaper and a half pint of Tetley's.) We were adventurous: Manarfon to Tenby in the south, Fishguard to Strumble Head Lighthouse up by the border with Ceredigion, and a fair percentage of the coast and estuary in between. We must have undertaken a few dozen expeditions that spring and summer, the exact routes for most of which, inevitably, I now struggle to trace on a map.

But the journey from Porthgain to Abercastle is one of the exceptions; there is no fog around that memory. I think -- in fact I know -- that it was her favourite walk -- and mine. I've been fascinated by Porthgain ever since that first trip with my parents when I was seven or eight. The place has a post-industrial, almost Audenian, romance to it, with its slipway and ruined quarries and the faint remains of a narrow-gauge railway. And then those four miles of paradise up the coast to Abercastle: the wild seaside flowers, and the little waterfall pouring into the ocean, and the banks of gorse smelling of coconuts and riddled with spiders.

`It's not suffering in and of itself that makes me doubt the existence of a loving God,' Paula announced one afternoon, as we rounded the hill east of Porthgain harbour, `it's the everyday nastiness of life. I can sort of see how the grand tragedies -- my situation, for example, or a young mother who gets diagnosed with brain cancer, or her children get killed in a car crash -- I can sort of see how experiences like that might elevate a person's soul -- although of course you would then ask what kind of supreme being would need to resort to those methods. But what about toothache or tapeworms or dandruff? Why create any of that stuff?'

`I think you might have a point there,' I said. `Forget Darwin and Galileo. How can one reconcile Christianity with the existence of Esther Rantzen?'

That got a chuckle. Then she talked a bit more about the kinds of thoughts she had been thinking.

`It's difficult to reconcile the prognosis they've given me with how I'm feeling right now. I don't feel like I'm dying.' Then she spoke about feeling afraid: `I know I should say that what really bothers me is the effect all this is going to have on Ellie and the others I'll leave behind. And of course I do worry about that, but that's not foremost in my mind. It's lights out, journeys end, ``Good night, Vienna'', timeless oblivion. Not being here, not being anywhere.'

As the outline of Abercastle became more discernible, I asked Paula if she would like to visit an ancient monument. She said she would; it was only three hundred yards from the Coastal Path. We stood on the grass with our toecaps touching the wire fence separating the monument from the byway. Unless you know what you're looking for, it just looks like a pile of boulders, about the same dimensions as a Land Rover Discovery. It acquired the name which appears on maps today, Carreg Samson, during the Middle Ages, long, long after it was built. It was likely built, I explained to Paula, around the same time that the Egyptians were putting up their pyramids. One book says there are probably hundreds of bodies buried underneath it. She closed her eyes and began to recite something from memory:

\begin{quote}
    But the dead know nothing, and have nothing left to hope for; but the memory of them is lost. All their love and their hate and their envy have already perished, and they have no more share forever in anything that is done under the sun.
\end{quote}

She told me it was from the Bible, but she couldn't remember which book; it didn't sound very biblical to me.

Abercastle was barely half a mile further down the coast. We sat on a bench overlooking its rocky beach. We were waiting for George to appear in the car park. Then Paula began talking about Tolstoy's \textit{Confession}. She had first read it in the lower sixth; one of the girls had brought a copy into the dorm, and it had caused an unlikely sensation as it made the rounds. `I suppose we were a very serious bunch, but there wasn't much to do in the evenings,' she said with a wry smile. `You made your own entertainment.' A slim volume, especially by Russian standards; there are probably crisp packets with a higher word count in that part of the world. The author asks himself whether life can have any meaning given the total annihilation to which we are all heading -- whether, in fact, life can be bearable at all. He wrote it in response to a midlife crisis, but it dovetails with teenage angst rather well. `It ends with a description of a dream he had had recently,' she said. `He's conscious of a void below him and a light above him, and he's floating between the two, but he doesn't understand what's keeping him afloat. He panics, and is actually about to drop into the darkness, until a voice tells him that, so long as he keeps his eys on the light, no harm can come to him, and so that's what he does. That image has stayed with me ever since, and it definitely describes where I am right now.'

`You've shown a lot of yourself to me today,' I said. `I know there's a school of thought that says you should never show yourself as you truly are, that if you show yourself naked to your friend, he cannot help losing respect for you. But I want to know that I reject that school of thought completely.'

`I'm afraid, old thing, that I have absolutely no desire to see you naked.'

Then we both dissolved into laughter.

\section{}

It finally happened on 10 April 1994, a Sunday, the night of the network premiere of \textit{Goodfellas} on Channel 4. I remember that because we watched it together -- Ellie, Paula and myself -- in the farmhouse living room. I was really excited; I had heard a lot about the film; I had been too young to see it when it came out in the cinema. Ellie opened a couple of bottles of wine -- Chablis, unfortunately -- I teased her that the occasion surely cried out for a robust Italian red. Paula speculated (with a broad smile) on the volume of bloodshed we were about to witness; but then, just after that scene where Billy Batts gets repeatedly perforated with a carving knife, she said that it was past her bedtime, and disappeared upstairs. Ellie moved off the armchair to sit next to me on the sofa. Then, about half an hour later -- I think it was during that tense last meeting between Ray Liotta and Robert de Niro in the diner -- she said it would be good if I were to visit her bedroom in a few minutes.

I worked up the courage and pushed open the door. Only a small bedside lamp was on, giving a cosy glow. She was lying in bed with the duvet pulled up to her chin, like a little girl waiting to be tucked in. I hovered for a few moments by the side of the bed.

`Come here,' she said, and pushed the covers to one side.

She was as naked as the day she was born. Her breasts were larger and whiter than I had pictured them. I realised that the familiar caramel of her face and arms must have come from her fondness for team games and middle-distance running; the rest of her was so pale. Her shoulders were thick and sinewy, as if they belonged to a boy in his late teens; that wasn't my thing, but neither did it put me off. She had a recent French bikini wax, as was fashionable in those days. The carpet matched the curtains.

I took my clothes off and lay down next to her.

`Can I kiss you?' I asked.

`Just a little bit.'

Her lips tasted of booze. She opened her eyes very wide when I entered her, but it was out of simple discomfort, and not any kind of epiphany of ecstasy. A couple of minutes into it, I got the feeling that, while she was trying her best to make me feel comfortable and to encourage me to continue, she wasn't having a good time; so I climaxed as quickly as I could. There was an attempt at cuddling afterwards, but I found myself lost for words, and I couldn't bear to spoil what for me was a sacred moment by talking about the weather or her last meal. So I told her I would leave her in peace, and got dressed. She kissed me on the cheek before I left.

I wandered down to the kitchen. I noticed a 375 ml bottle of vodka on one of the window sills, half-hidden behind a vase. There wasn't much left. I poured myself a shot and drank it, and then another. That was the last of it. Then I went to bed.

I didn't see \textit{Goodfellas} all the way through until I was thirty-five.

\section{}

After that night, we did it two or three times a week, depending on how often she was home. We largely followed the same script each time. Then around the beginning of July she informed that she was pregnant, and thus my services would no longer be required. (I'm sure she must have put it more gently than that, but that's how I remember the conversation unfolding.) I was overjoyed, of course, but a part of me was also crushed by the news. For all the strangeness, sleeping with her meant the world to me.

Ellie mentioned years later that her first trimester had been really horrible, but at the time I had no idea. In my defense, she kept me at arm's length throughout the pregnancy; I was lucky if the three of us ate together even once week. By September, Paula's decline had accelerated visibly; she was struggling to grip a pint glass properly, even with her good hand, and she was starting to slur her words. By Christmas, her walking had become unsteady. At a lunch with a dozen or so of her wider friend group, she remarked that she was now at a stage of life where difficulty walking was a sign of decrepitude, rather than a vigorous romantic life, and brought the house down.
