\proseepigraph{I have learned one thing: not to look down\\So much upon the damned.}\blindfootnote{The epigraph is taken from Prof Hill's `Ovid in the Third Reich'. The story itself may be read as a lyrical reflection on the philosophical problems raised by \textit{Sex and the City} -- both the television series and the subsequent films -- although other interpretations are of course possible.}

It was only in the autumn of 1941 that Magda Goebbels decided to defect. Contrary to what you might have heard, no hostilities broke out in September 1939. Field Marshal von Hindenburg, that redoubtable old cabbage, clung onto life until 1940, while Reinhard Heydrich was involved in a particularly unfortunate flying accident in which he lost both eyeballs, both testicles and both arms. When the crisis came, cooler heads prevailed. Danzig was hardly worth a global conflagration, and the F\"uhrer realised that, even within her current borders, Germany's resources were such that she was destined to master Europe forever. The Jewish question could be dealt with through emigration and the dilutions of intermarriage; violence, he said, would only draw unwanted attention. World domination could be achieved without the risks and hardships of war.

But by 1941 Magda Goebbels had made up her mind. Josef's homoerotic preoccupation with the boss was a powerful anaphrodisiac. His affairs with actresses did not help, but these were carnal, not spiritual infidelities. Even as she entered her fortieth year, she remained a passionate woman, with a passionate woman's need to be the sole object of her husband's worship.

It was her friends, of course, who gave her the nerve to go through with it. Indeed, by Easter 1942 she had built up a whole clandestine organisation. By then, Eva Braun had realised that relations with an absolute ruler were less satisfactory than is commonly imagined, that Adolf would not marry her in the forseeable future, and that he would never give her a child. Emmy G\"oring, too, was tired of being mounted by a beachball wrapped in tinsel. Even Margarete Himmler -- who even among reproving German housewives enjoyed a reputation as a humourless scold -- ended up joining the group. The son whom Heinrich had fathered with his secretary proved too much to bear.

Therefore, early in the morning of Easter Monday, while their husbands' hangers-on were still nursing their hangovers, the four women made their way, separately, to the American embassy in Berlin, as agreed in advance. Leland Morris, the charg\'e d'affairs, was still in his dressing gown when he was told the news. Roosevelt's chief of staff had to wake him in the middle of the night. (In Washington, it was barely half-past three.) But almost immediately it was welcomed as a wonderful surprise, a magnificent propaganda victory in a conflict yet to ignite. The women were to be housed wherever in the contiguous United States (within reason) they might like, with generous allowances and full immunity for any crimes to which they may have been a party in their former lives. They were to be moulded into exemplars of the new Germany -- the one the Americans hoped to build once Hitler and his circle had been taken out of the picture.

\divsep

The women chose to settle in New York. The shops and restaurants there were probably the best in the country, and, besides, it had a significant German emigr\'e community -- one which was especially sympathetic to political exiles. There was some talk about them getting jobs, but -- apart from Emmy G\"oring, who was to make a half-hearted stab at a Broadway career -- none of them really had any marketable skills. The question soon proved to be redundant. The United States Government decided to gift the women a lump sum for their defection, not enough to live on forever, but a decent pile of cash. That was the spark for the next thing that happened. The tinder was that Margarete Himmler, while she was still in Germany, had overheard several conversations between her husband and Ernst Kaltenbrunner, his most senior lieutenant following Heydrich's tragic accident. (She was expected to serve coffee at all the meetings her husband took at home.) Using her eavesdropping as insider information, she put the group's money into what must have seemed like a wild hodgepodge of investments. There was a company which made cyanide-based pesticides, an engineering firm which produced machinery for melting down smaller items -- gold watches, gold rings, gold teeth, etc -- into more fungible bars and bullion. There was even a company selling super-hot industrial-size ovens; the founders used to boast that they could reduce a saddle of lamb to a handful of ashes in under a minute. These ventures soon paid out astonishing dividends. Within a few months, all four women were fabulously wealthy.

Not having much in common with the rest of the city, they booked out the top floor of the Waldorf Astoria in perpetuity and made it their home. They jokingly named their little gang the Faith and Beauty Society -- a nod to the section of the League of German Maidens for teenage girls in full bloom.\smallmarginnote{BDM-Werk Glaube und Schönheit} No longer burdened with the need to earn a living, each was now free to pursue her real passion. Emmy G\"oring, as I said, gave acting another go, while Eva Braun soon earned the quiet disgust of the others. There was a coloured gentleman -- to use the parlance of the time -- who operated the elevator she used to frequent. She invited him back to her room for a drink once he had finished his shift, and they became lovers that same night. He was a quiet, thoughtful, awkward man. She gave him three beautiful, lively daughters in as many years -- she thoroughly enjoyed his company -- who varied in complexion from butterscotch to coltsfoot rock.

But the real shock was Margarete Himmler. That coldest of fish became a determined hedonist. No wordly pleasure was permitted to float past untasted. Falling in love with the cuisine of her adopted city, she installed private kitchen next to her room, with a giant salami slicer so that the prodigious quantities pastrami and ox tongue she required were always freshly cut, and a double fridge just for cheesecake. She soon acquired a fuller figure -- plump, but never corpulent -- and a taste for creme de menthe, cointreau and sweetened negronis. But her great indulgence was, of course, the sons of Adam. No one can fault a newly-single woman for needing company, but this lady wanted a whole battalion. And you would be amazed by the number of virile, handsome young men for whom sleeping with a Nazi princess made it into the top five items on his bucket list. In the end, with the help of a grey-whiskered emigr\'e accountant by the name of Herr Klamm,\smallmarginnote{\frakturish Das Schloß} she had to introduce a kind of discreet ticket system. Her waking life was divided into twenty-minute sections -- her lovers were almost exclusively in their early twenties -- at least five of which per day were allocated to gentleman callers. In the evenings, she liked to open the curtains and switch off the lights; on the top floor, they were too high up for anyone to see anything anyway. Clasping another hunk of stripling, satisfied flesh to her chest, she often wondered how this compared to the paradise which National Socialism had once promised her.

\divsep

Of course, war did eventually break out. Hitler, prodded forward by his own ideology as much as strategic necessity, attacked the Soviet Union in the summer of 1949. Despite some early successes, the largest invasion in world history did not go as planned. The Soviets had put the decade following the signing of the Molotov-Ribbentrop Pact to good use: more soldiers, frenetic investment in the steelworks and sweatshops of the Urals, lots and lots of tanks. Army Group North was ground down in urban warfare in and around Pskov. Army Group South succeeded in capturing most of Kiev, but failed, despite several desperate attempts, to cross the Dnieper. Army Group Centre seemed to have more luck -- they smashed through Minsk and Smolensk with such apparent ease that the Soviets reverted to their old custom of shooting their own commanders -- until Guderian's panzers were routed in an epic tank battle fifty kilometres west of Vyazma. Once the front line had stabilised, Marshal Tukhachevsky -- who had only avoided liquidation in the purges of the 1930s by the skin of his teeth -- put his Theory of Deep Operations to good use. By the spring of 1952, Stalin was sat at his desk with the Königgrätzer Marsch playing on a loop, pissing himself laughing and using Hitler's skull as an ashtray. But all this, I must stress, was years in the future; none of the women we are concerning ourselves with now would have the opportunity either to mourn or to celebrate the outcome of the Great Patriotic War.

Simone de Beauvoir published \textit{The Second Sex} in 1945; the unexpected peace had lent her writing a strange fluency. As its author had anticipated -- she had spent half her life being stuffed with some of the most exalted accolades the \textit{\'ecoles sup\'erieures} could bestow -- the book quickly made headlines, and not just in France and her colonies but across the English-speaking world as well. Manhattan in particular was a Mecca of Beauvoirites and Beauvoiristas, pumping out evangelists to enlighten the rest of the continent. Magda Goebbels, never one to miss an up-and-coming political movement, was one of the first New Yorkers to obtain a copy.

She claimed to have devoured the book almost as soon as she had obtained it, but in fact it sat unopened on her nightstand for several months -- until one rainy Wednesday when all the good cafes were closed for lack customers, and it was too early to pour herself a proper drink. She had gushed over the book at parties, but, now that she was actually turning the pages, her feelings were more mixed. The distinction between gender and biological sex struck her as pure casuistry. She liked, however, the passages about a woman's obligation to walk her own path, and the necessity of jettisoning, in keeping to that path, any inherited notions of good and evil. That, it struck Magda Goebbels, was actually quite close to her own philosophy, although she had never felt the need herself to spell it out in those terms. Of course there had been sacrifices; it bothered her, in particular, that she was not likely to see any of her children ever again. But, if such sacrifices were the price of her self-actualisation, it was worth it. She had done the right thing. In fact -- she only realised this just now -- she had \emph{always} done the right thing: leaving her first husband, joining the party, marrying Josef, hosting such wonderful state banquets, running away, making a new life overseas. That was her last thought before the bullet entered the back of her skull, spattering blood and brain matter over the pages.

The same Abwehr agent who killed Magda Goebbels moved quietly and efficiently through the other rooms in the floor, despatching the remaining women one by one. Eva Braun he filleted like a prize salmon, from between her legs to the crown of her head. Having incapacitated Margarete Himmler with a paralytic agent, he then fed her, feet first, into her beloved salami slicer. Last of all, he strung up Emmy G\"oring with good old-fashioned piano wire.

He took no pleasure in any of these horrific killings. Quite the opposite: he was tormented by the memory of them for the rest of his life. But he took some comfort from the fact -- as he exited the building by the fire escape, as he climbed on board the U-boat waiting to ferry him home, as Admiral Canaris pinned the medal on his uniform, indeed every time thereafter that he closed his eyes to sleep -- that he was merely implementing the guidelines which more exalted persons had given him.
